%------------------------------------------------------------------------------
\chapter{Esettanulmány}
%------------------------------------------------------------------------------
\section{A  feladat összefoglalása}
%------------------------------------------------------------------------------
Az önálló munkám során a Méréstechnika és Információs Rendszerek Tanszék Hibatűrő Rendszerek Kutatócsoport (továbbiakban: ftsrg) által fejlesztett Gamma Statechart Composition Framework-öt (továbbiakban: Gamma) vettem kiindulási pontnak.
A feladat során a már létező rendszert modelleztem különféle módszerek segítségével, ezáltal bemutatva a két modellezési nyelv sajátosságait és különbségeit.

\section{Gamma Statechart Composition Framework}
A Gamma Statechart Composition Framework komponens-alapú reaktív rendszerek modellezésére, verifikációjára és kódgenerálására szolgáló eszközkészlet.
A keretrendszer a nyílt-forrsákódú Yakindu állapotdiagram modellező eszközre épít és ehhez szolgáltat extra modellezési réteget a hálózatban kommunikáló komponensek leírására.\cite{molnar2018gamma}

Egyedi állapotdiagrammok vagy összetett állapotdiagram hálózatok validációját és verifikációját az UPPAAL nevű - időzített autómaták számára kifejlesztett modell ellenőrző - szoftver végzi egy automatikus modelltranszformáció segítségével.

Ha a teljes modell elkészült a tervezők használhatják a keretrendszer kód generálási funkcióját, ami képes futtatható Java kódot generálni a teljes rendszerhez.
